% !TeX root = main.tex

\hypertarget{antiderivatives}{%
\section{Antiderivatives}\label{antiderivatives}}

\begin{definition}

A function \(F\) is an antiderivative of the function \(f\) if
\[F'(x)=f(x)\] for all \(x\) in the domain of \(f\).

\end{definition}

\hypertarget{the-most-general-form-of-an-antiderivative}{%
\subsection{The most general form of an
antiderivative}\label{the-most-general-form-of-an-antiderivative}}

Let \(F\) be an antiderivative of \(f\) over an interval \(I\). Then, by
the mean value theorem, for each constant \(C\), the function \(F(x)+C\)
is also an antiderivative of \(f\) over \(I\); if \(G\) is an
antiderivative of \(f\) over \(I\), there is a constant \(C\) for which
\(G(x)=F(x)+C\) over \(I\). In other words, the most general form of the
antiderivative of \(f\) over \(I\) is \(F(x)+C\).

% \begin{example}

% For each of the following functions, find all antiderivatives.

% \begin{enumerate}
% \item
%   \(f(x)=3x^2\)
% \item
%   \(f(x)=\dfrac{1}{x}\)
% \item
%   \(f(x)=\cos x\)
% \end{enumerate}

% \end{example}

\hypertarget{rules-of-antiderivatives}{%
\subsection{Rules of Antiderivatives}\label{rules-of-antiderivatives}}

\begin{longtable}[]{@{}cc@{}}
\toprule()
Function & An antiderivative \\
\midrule()
\endhead
\(af+bg\) & \(aF+bG\) \\
\(x^n\), \(n\neq -1\) & \(\frac{x^{n+1}}{n+1}\) \\
\(\sin x\) & \(-\cos x\) \\
\(\cos x\) & \(\sin x\) \\
\(\sec^2x\) & \(\tan x\) \\
\(\sec x\tan x\) & \(\sec x\) \\
\bottomrule()
\end{longtable}

In the above table, \(F\) and \(G\) are antiderivative of \(f\) and
\(g\) respectively, \(a\) and \(b\) are arbitrary constants.

\begin{example}

Find the most general antiderivative for each functions.

\begin{enumerate}
\item
  \(f(x)=5x^3-7x^2+3x+4\)
\item
  \(f(x)=\dfrac{x^2+4\sqrt[3]{x}}{x}\)
\item
  \(f(x)=\dfrac{4}{1+x^2}\)
\item
  \(f(x)=\tan x\cos x\)
\end{enumerate}

\end{example}

\hypertarget{antiderivative-with-initial-value}{%
\subsection{Antiderivative with initial
value}\label{antiderivative-with-initial-value}}

The most general antiderivative defines a family of curves. If a point
is given, then there will be specific antiderivative. The problem of
finding a function \(y\) that satisfies a differential equation
\(\dfrac{dy}{dx}=f(x)\) with the additional condition \(y(x_0)=y_0\) is
known as an initial-value problem. The condition \$ y(x\_0)=y\_0\$ is
known as an initial condition.

\begin{example}

Solve the initial-value problem
\[\dfrac{dy}{dx}=\sin x,\quad\text{with}\quad y(0)=5.\]

\end{example}
\vspace*{6\baselineskip}

\begin{example}

Find an equation for the function \(y\) that satisfies the following
conditions \[\dfrac{dy}{dx}=3x^{-2},\quad\text{and}\quad y(1)=2.\]

\end{example}
\vspace*{6\baselineskip}

\begin{example}

Find an equation for the function \(f\) that satisfies the following
conditions \[f''(x)=\sin x,\quad f'(0)=1, \text{and}\quad f(\pi)=0.\]

\end{example}
\vspace*{6\baselineskip}

\hypertarget{application}{%
\subsection{Application}\label{application}}

\begin{example}

A car is traveling at the rate of \(88 \text{ ft}/\text{s}\) when the
brakes are applied. The car begins decelerating at a constant rate of
\(15 \text{ ft}/\text{s}^2.\)

\begin{enumerate}
\item
  How many seconds elapse before the car stops?
\item
  How far does the car travel during that time?
\end{enumerate}

\end{example}

\hypertarget{practice}{%
\subsection{Practice}\label{practice}}

\begin{exercise}

Find the antiderivative \(F(x)\) of each function \(f(x).\)

\begin{enumerate}
\item
  \(f(x)=5x^4+4x^5\)
\item
  \(f(x)=sec^2(x)+1\)
\item
  \(f(x)=2\sin x-\cos x\)
\item
  \(f(x)=\frac{x+1}{\sqrt{x}}\)
\end{enumerate}

\end{exercise}

\begin{exercise}

Solve the initial value problem.

\begin{enumerate}
\item
  \(f'(x)=x^{-3},\quad f(1)=1\)
\item
  \(f'(x)=\sqrt{x}+x^2,\quad f(0)=2\)
\item
  \(f''(x)=x^2+2,\quad f'(0)=1,\quad f(1)=2\)
\end{enumerate}

\end{exercise}

\begin{exercise}

A car is being driven at a rate of \(40\) mph when the brakes are
applied. The car decelerates at a constant rate of \(10\) ft/sec2. How
long before the car stops?

\end{exercise}

