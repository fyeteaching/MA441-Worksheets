% !TeX root = main.tex

\hypertarget{derivatives-of-trigonometric-functions}{%
\section{Derivatives of Trigonometric
Functions}\label{derivatives-of-trigonometric-functions}}

\hypertarget{the-derivative-of-the-sine-function}{%
\subsection{The Derivative of the Sine
Function}\label{the-derivative-of-the-sine-function}}

Recall the formula of sum of angles 
\[
\begin{aligned}
  \sin(x+h)=&\sin x\cos h+\cos x\sin h\\[0.5em]
  \cos(x+h)=&\cos x\cos h-\sin x\sin h.
\end{aligned}
\] Then for any \(x\), we have \[
\begin{aligned}
  \dfrac{\mathrm{d}}{\mathrm{d} x}(\sin x)=&\lim\limits_{h\to 0}\dfrac{\sin(x+h)-\sin x}{h}\\
  =&\lim\limits_{h\to 0}\dfrac{\sin x\cos h+\cos x\sin h-\sin x}{h}\\
  =&\lim\limits_{h\to 0}\dfrac{\sin x(\cos h-1)}{h}+\lim\limits_{h\to 0}\dfrac{\cos x\sin h}{h}\\
  =&\sin x\lim\limits_{h\to 0}\dfrac{\cos h-1}{h}+\cos x\lim\limits_{h\to 0}\dfrac{\sin h}{h}\\
  =&\sin x\cdot 0 +\cos x\cdot 1\\
  =&\cos x.
\end{aligned}
\]

\begin{example}

Find the derivative of \(f(x)=x^2\sin x\).

\end{example}
\vspace*{6\baselineskip}

\hypertarget{derivatives-of-other-trigonometric-functions}{%
\subsection{Derivatives of Other Trigonometric
Functions}\label{derivatives-of-other-trigonometric-functions}}

Using the sum angle formula for cosine and limit laws of quotients, we
obtain the following formulas of derivative of trigonometric functions.
\[
\begin{aligned}
\dfrac{\mathrm{d}}{\mathrm{d} x}\sin x &= \cos x \\
\dfrac{\mathrm{d}}{\mathrm{d} x}\cos x &= -\sin x\\
\dfrac{\mathrm{d}}{\mathrm{d} x}\tan x &= \sec^2x\\
\end{aligned}
\qquad
\begin{aligned}
\dfrac{\mathrm{d}}{\mathrm{d} x}\csc x &= -\csc x \cot x \\
\dfrac{\mathrm{d}}{\mathrm{d} x}\sec x &= \sec x\tan x\\
\dfrac{\mathrm{d}}{\mathrm{d} x}\cot x &= -\csc^2x\\
\end{aligned}
\]

\begin{example}

Find the derivative of \(f(x)=\cos x\sin x\).

\end{example}
\vspace*{6\baselineskip}

\begin{example}

Find the derivative of \(f(x)=\dfrac{\cos x-1}{\sin x+1}\).

\end{example}
\vspace*{6\baselineskip}

\begin{example}

Find the derivative of \(f(x)=2x\tan x-3\cot x\).

\end{example}
\vspace*{6\baselineskip}

\begin{example}

Find the derivative of \(f(x)=\sec^2x-x\csc x\).

\end{example}
\vspace*{6\baselineskip}

\begin{example}

Find the 59-th derivative
\(\dfrac{\mathrm{d}^{59}}{\mathrm{d}x^{59}}(\sin x)\).

\end{example}
\vspace*{6\baselineskip}

\subsection{Practice}

\begin{exercise}

  Find the derivative of \(f(x)=\dfrac{x^3+x}{\sin x}\).
  
  \end{exercise}
  \vspace*{6\baselineskip}
  
\begin{exercise}

Find the derivative of \(f(x)=\dfrac{\sin x+\cos x}{\sin x-\cos x}\).

\end{exercise}
\vspace*{6\baselineskip}

\begin{exercise}

Find the derivatives of the function \(f(x)=3\tan x+\sqrt{x^5}\).

\end{exercise}
\vspace*{6\baselineskip}

\begin{exercise}

Find the derivative of \(f(x)=(x+\tan x)(\sec x + x^2)\).

\end{exercise}
\vspace*{6\baselineskip}

\begin{exercise}

Find the 3-th derivative
\(\dfrac{\mathrm{d}^{3}}{\mathrm{d}x^{3}}(\tan x)\).

\end{exercise}
\vspace*{6\baselineskip}

