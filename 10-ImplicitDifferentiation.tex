% !TeX root = main.tex

\hypertarget{implicit-differentiation}{%
\section{Implicit Differentiation}\label{implicit-differentiation}}

As an application of the chain rule, the technique of \textbf{implicit
differentiation} allows us to find the derivative of an implicitly
defined function without ever solving for the function explicitly.

\textbf{Problem-Solving Strategy: Implicit Differentiation}

\begin{enumerate}[sepno]
\item
  Take the derivative of (or differentiate)both sides of the equation.
  Keep in mind that \(y\) is a function of \(x\).
\item
  Solve for \(y'\) (or \(\dfrac{\mathrm{d}y}{\mathrm{d}x}\)) from the
  resulting equation.
\end{enumerate}

\begin{example}

Find \(\dfrac{\mathrm{d}y}{\mathrm{d}x}\) where \(y\) is a function of
\(x\) defined by the equation \(x^3+y^2=1\).

\end{example}
\vspace*{6\baselineskip}

\begin{example}

Find \(\dfrac{\mathrm{d}y}{\mathrm{d}x}\) where \(y\) is a function of
\(x\) defined by the equation \(2\sin x-1=\cos y\).

\end{example}
\vspace*{6\baselineskip}

\begin{example}

Find \(\dfrac{\mathrm{d}y}{\mathrm{d}x}\) for the function \(y\)
implicitly defined by the equation \(x^3\sin y+\tan y=3x+2\).

\end{example}
\vspace*{6\baselineskip}

\begin{example}

Find \(\dfrac{\mathrm{d}^2y}{\mathrm{d}x^2}\) given that \(y\) is
implicitly defined by \(x^2-y^2=15\).

\end{example}
\vspace*{6\baselineskip}

\begin{example}

Find the equations of lines tangent to the curve
\(2xy=\dfrac xy+\dfrac yx\) at the point (1,1).

\end{example}
\vspace*{6\baselineskip}

\begin{example}

The number of cars produced when \(x\) dollars is spent on labor and
\(y\) dollars is spent on capital invested by a manufacturer can be
modeled by the equation \(\displaystyle 40x^{\frac13} y^{\frac23}=480\).

\begin{enumerate}
\item
  Find a formula in terms of \(x\) and \(y\) for
  \(\displaystyle \frac{\mathrm{d}{y}}{\mathrm{d}{x}}\). \hspace{0pt}
\item
  Find the value of of
  \(\displaystyle \frac{\mathrm{d}{y}}{\mathrm{d}{x}}\) at the point
  (27,8).
\end{enumerate}

\end{example}

\begin{example}

Find all points \((x,y)\) on the graph of the equation
\(2x^2 + 8y^2= x + 4 y\) such that \(x>0\), \(y>0\) and at where the
tangent line is

\begin{enumerate}
\item
  horizontal
\item
  vertical
\end{enumerate}

\end{example}

\subsection{Practice}

\begin{exercise}

Find \(y'\) where \(y\) is a function of \(x\) defined implicitly by the
equation \(x^2-xy+y^2=1\).

\end{exercise}
\vspace*{6\baselineskip}

\begin{exercise}

Find \(\dfrac{\mathrm{d}y}{\mathrm{d}x}\) for \(y\) defined implicitly
by the equation \(x^2\cos y+y^2=3x+1\).

\end{exercise}
\vspace*{6\baselineskip}

\begin{exercise}

Find the \(\dfrac{\mathrm{d}^2y}{\mathrm{d}x^2}\) where \(y\) is a
function defined implicitly by the equation \(x^3+y^2=3x+1\).

\end{exercise}
\vspace*{6\baselineskip}

\begin{exercise}

Find the equation of the line tangent to the graph of \(y^3+x^3 - 3xy=0\)
at the point \((\frac32,\frac32)\).

\end{exercise}
\vspace*{6\baselineskip}

\begin{exercise}

Find all points \((x,y)\) on the graph of the equation
\(2x^2 + 8y^2= x + 4 y\) such that \(x>0\), \(y>0\) and at where the
tangent line is

\begin{enumerate}
\item
  horizontal
\item
  vertical
\end{enumerate}

\end{exercise}

