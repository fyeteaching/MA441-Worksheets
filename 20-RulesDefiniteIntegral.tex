% !TeX root = main.tex

\hypertarget{definite-integrals}{%
\section{Definite Integrals}\label{definite-integrals}}

\hypertarget{definite-integrals-1}{%
\subsection{Definite Integrals}\label{definite-integrals-1}}

\begin{definition}

Let \(f(x)\) be an integrable function, particularly, a continuous
function, defined over an interval \([a,b]\). The \textbf{definite
integral} of \(f\) from \(a\) to \(b\) is defined as
\[\int^b_af(x)\mathrm{d}x=\lim\limits_{n\to\infty}\sum\limits_{i=1}^nf(x^*_i)\Delta x.\]
The symbol \(\int\) is called an integral sign. The numbers \(a\) and
\(b\) are called the \textbf{upper and lower limits of integration}. The
function \(f(x)\) is called the \textbf{integrand}. The
\textbf{differential} \(\mathrm{d} x\) is called the \textbf{variable of
integration}.

\end{definition}

Like the index in a sigma notation, the variable of integration is a
dummy variable, which mean you may use any other letter instead of \(x\)
to write the integral.

\textbf{Fact:} The definite integral \(\int_a^bf(x)\mathrm{d}x\) is the
signed area under the curve \(f\) over the interval \([a, b]\).

\begin{example}

Find the definite integral \(\int_a^bc\mathrm{d}x\) using a Riemann sum
and a graph.

\end{example}
\vspace*{6\baselineskip}

\begin{example}

Find the definite integral \(\int_a^bx\mathrm{d}x\) using a Riemann sum
and a graph.

\end{example}
\vspace*{6\baselineskip}

\begin{example}

Find the definite integral \(\int_{a}^bx^2\mathrm{d}x\) using a Riemann
sum.

\end{example}
\vspace*{6\baselineskip}

\begin{example}

Find the definite integral \(\int_0^1\sqrt{1-x^2}\mathrm{d}x\) using a
graph.

\end{example}
\vspace*{6\baselineskip}

\begin{example}

Express the limit as an integral and evaluate
\[\lim\limits_{x\to \infty}\frac{1}{n}\sum_{i=1}^n\frac{i^2}{n^2}\]

\end{example}
\vspace*{6\baselineskip}

\hypertarget{signed-area}{%
\subsection{Signed Area}\label{signed-area}}

\begin{definition}

\textbf{(Signed Area)} Let \(f(x)\) be an integrable function defined on
an interval \([a,b]\). Let \(A_1\) represent the area between \(f(x)\)
and the \(x\)-axis that lies above the axis and let \(A_2\) represent
the area between \(f(x)\) and the \(x\)-axis that lies below the axis.
Then, the~(net) signed area~between \(f(x)\) and the \(x\)-axis is given
by \[\int^b_af(x)\,dx=A_1 - A_2.\] The total area between \(f(x)\) and the
\(x\)-axis is given by \[\int^b_a|f(x)|\,dx=A_1+A_2.\]

\end{definition}

\begin{example}

Find the total area between the function \(f(x)=x-2\) and the \(x\)-axis
over the interval \([0,3]\).

\end{example}
\vspace*{6\baselineskip}

\hypertarget{rules-of-definite-integrals}{%
\subsection{Rules of Definite
Integrals}\label{rules-of-definite-integrals}}

\begin{enumerate}[sepno]
\item
  \[\int^a_af(x)\,\mathrm{d}x=0\] If the limits of integration are the
  same, the integral is just a line and contains no area.
\item
  \[\int^a_bf(x)\,\mathrm{d}x= - \int^b_af(x)\,\mathrm{d}x\] If the limits
  are reversed, then place a negative sign in front of the integral.
\item
  \[\int^b_a[f(x)+g(x)]\,\mathrm{d}x=\int^b_af(x)\,\mathrm{d}x+\int^b_ag(x)\,\mathrm{d}x\]
  The integral of a sum is the sum of the integrals.
\item
  \[\int^b_a[f(x) - g(x)]\,\mathrm{d}x=\int^b_af(x)\,\mathrm{d}x - \int^b_ag(x)\,\mathrm{d}x\]
  The integral of a difference is the difference of the integrals
\item
  \[\int^b_acf(x)\,\mathrm{d}x=c\int^b_af(x)\,\mathrm{d}x\] for constant
  \(c\). The integral of the product of a constant and a function is
  equal to the constant multiplied by the integral of the function.
\item
  \[\int^b_af(x)\,\mathrm{d}x=\int^c_af(x)\,\mathrm{d}x+\int^b_cf(x)\,\mathrm{d}x\]
  Although this formula normally applies when \(c\) is between \(a\) and
  \(b\), the formula holds for all values of \(a\), \(b\), and \(c\),
  provided \(f(x)\) is integrable on the largest interval.
\end{enumerate}

\begin{example}

If it is known that \(\displaystyle \int^5_1f(x)\,\mathrm{d}x= - 3\) and
\(\displaystyle \int^5_2f(x)\,\mathrm{d}x=4\), find the value of
\(\displaystyle \int^2_1f(x)\,\mathrm{d}x.\)

\end{example}
\vspace*{6\baselineskip}

\begin{theorem}

\textbf{Comparison Theorem:}

\begin{enumerate}[sepno]
\item
  If \(f(x) \ge 0\) for \(a \le x \le b\), then \[\int^b_af(x)\,\mathrm{d}x \ge 0.\]
\item
  If \(f(x) \ge g(x)\) for \(a \le x \le b\), then
  \[\int^b_af(x)\,\mathrm{d}x \ge \int^b_ag(x)\,\mathrm{d}x.\]
\item
  If \(m\) and \(M\) are constants such that \(m \le f(x) \le M\) for \(a \le x \le b\),
  then \[m(b - a) \le \int^b_af(x)\,\mathrm{d}x \le M(b - a).\]
\end{enumerate}

\end{theorem}

\begin{example}

Determine the sign of
\(\int_0^1(\sqrt{1+x^2}-\sqrt{1+x^4})\mathrm{d} x\)

\end{example}
\vspace*{6\baselineskip}

\subsection{Practice}

\begin{exercise}

Using geometry to evaluate the integral
\[\displaystyle \int^3_{ - 3}(3 - |x|)\,dx.\]

\end{exercise}
\vspace*{6\baselineskip}

\begin{exercise}

Express the limit as integrals
\[\displaystyle \lim_{n\to\infty}\sum_{i=1}^n\sin^2(2\pi x^*_i)\Delta x\quad \text{over}\quad [0,1].\]

\end{exercise}
\vspace*{6\baselineskip}

\begin{exercise}

Using integral to evaluate the limit
\[\displaystyle \lim\limits_{n\to \infty}\frac{1}{n}\sum_{i=1}^n\frac{i - 1}{n}\]

\end{exercise}
\vspace*{6\baselineskip}

\begin{exercise}

Suppose that \(\displaystyle \int^4_0f(x)\,dx=5\) and
\(\displaystyle \int^2_0f(x)\,dx= - 3\), and
\(\displaystyle \int^4_0g(x)\,dx= - 1\) and
\(\displaystyle \int^2_0g(x)\,dx=2\) Compute the following integrals.

\begin{enumerate}
\item
  \(\displaystyle \int^4_0(3f(x) - 4g(x))\,dx\)
\item
  \(\displaystyle \int^4_2(2f(x)+3g(x))\,dx\)
\end{enumerate}

\end{exercise}

\begin{exercise}

Show that
\(\displaystyle \int^{\pi/4}_{ - \pi/4}\cos t\,dt \ge \pi\sqrt{2}/4\).

\end{exercise}

