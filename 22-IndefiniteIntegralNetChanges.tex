% !TeX root = main.tex

\hypertarget{indefinite-integrals-and-net-changes}{%
\section{Indefinite Integrals and Net
Changes}\label{indefinite-integrals-and-net-changes}}

\begin{definition}

Let \(F\) be an antiderivative of the function \(f\), that is
\(F'(x)=f(x)\). We call the most general anti-derivative of \(f\) is the
indefinite integral and denoted as \[\int{{f(x)\,dx}} = F(x) + c,\]
where \(c\) is any constant.

\end{definition}

\textbf{Formulas of Indefinite Integrals:}

\begin{itemize}
\item
  \(\displaystyle \int\left(af( x )+bg(x)\right)\,dx = a\int f( x )\,dx + b\int g( x )\,dx\)
\item
  \(\displaystyle \int x^n \,dx = \dfrac{x^{n+1}}{n+1} + C\), as long as
  \(n \ne -1\)
\item
  \(\displaystyle \int \cos(x) \, dx = \sin(x) + C\)
\item
  \(\displaystyle \int \sin(x) \, dx = - \cos(x) + C\)
\item
  \(\displaystyle \int \sec^2(x) \, dx = \tan(x) + C\)
\item
  \(\displaystyle \int \sec(x) \tan(x) \, dx = \sec(x) + C\)
\item
  \(\displaystyle \int \csc^2(x) \, dx = - \cot(x) + C\)
\item
  \(\displaystyle \int \csc(x)\cot(x) \, dx = -\csc(x) + C\)
\end{itemize}

\begin{example}

Evaluate the following indefinite integral.

\begin{enumerate}
\item
  \(\displaystyle\int (x^4 + 3x - 2)\,dx\)
\item
  \(\displaystyle\int\sqrt{x}(x^2 - 1)\,dx\)
\item
  \(\displaystyle\int (\sin x-\sec^2 x)\,dx\)
\end{enumerate}

\end{example}

\begin{definition}

The \textbf{net change} of a quantity is the integral of the rate of
change of that quantity. \[\text{Net change}=\int_a^b r(t)\,dt\] where
\(r(t)\) is the rate of change function.

Note the \textbf{total change} is
\[\text{Total change}=\int_a^b |r(t)|\,dt\] where \(r(t)\) is the rate of
change function.

\end{definition}

% \begin{example}

% Water is flowing into a tank at a rate of \(r(t)=3t^2-2t\) ft\(^3\)/min.
% How much water flows into the tank over the time interval 2 min. to 4
% min.?

In Physics, the net and total changes are known as the displacement and
distance. The distinction is between velocity and speed.

Displacement = Net change in position = integral of velocity.

Distance traveled Total = change in position = integral of speed.

% \end{example}
% \vspace*{6\baselineskip}

\begin{example}

A particle moves along a straight line. The velocity is observed as a
linear function \(v(t)=2t-6\) m/s from time \(t=0\) to time \(t=3\).

\begin{enumerate}
\item
  Find the net displacement of the particle.
\item
  Find the distance the particle traveled.
\end{enumerate}

\end{example}

\begin{definition}

Suppose that \(f\) is continuous over \([a,b]\). The \textbf{average
value of the function} of \(f\) over \([a, b]\) is defined as
\[f_{ave}=\dfrac{1}{b-a}\int_a^bf(x)dx.\]

\end{definition}

% \begin{example}

% Let \(f\) be a continuous function over \([a, b]\). Show that there is
% number \(c\) in \([a,b]\) such that
% \[f(c)=\dfrac{1}{b-a}\int_a^bf(x)dx.\]

% \end{example}
% \vspace*{6\baselineskip}

\begin{example}

Find the average of the function \(f(x)=x^2\) over \([-2, 2]\).

\end{example}
\vspace*{6\baselineskip}

\hypertarget{integration-of-even-and-odd-functions}{%
\subsection{Integration of Even and Odd
Functions}\label{integration-of-even-and-odd-functions}}

For continuous even functions \(f\), that is \(f(-x)=f(x)\),

\[\displaystyle \int^a_{-a}f(x)\,dx=2\int^a_0f(x)\,dx.\]

For continuous odd functions \(f\), that is \(f(-x)=-f(x),\)

\[\displaystyle \int^a_{-a}f(x)\,dx=0.\]

\begin{example}

Evaluate the following integrals.

\begin{enumerate}
\item
  \(\displaystyle \int^2_{-2}(3x^4+1)\,dx\)
\item
  \(\displaystyle \int_\pi^{-\pi} 6\sin x\, dx\)
\end{enumerate}

\end{example}

\subsection{Practice}

\begin{exercise}

Evaluate the indefinite integral

\begin{enumerate}
\item
  \(\displaystyle \int (\sqrt{x}-\frac{1}{\sqrt{x}})\,dx.\)
\item
  \(\displaystyle \int(\sin x-\cos x)\,dx\)
\end{enumerate}

\end{exercise}

\begin{exercise}

Suppose that a particle moves along a straight line with velocity
\(v(t)=4-2t,\) where \(0\leq t\leq 2\) (in meters per second).

Find the displacement at time t and the total distance traveled up to
\(t=2\).

\end{exercise}
\vspace*{6\baselineskip}


\begin{exercise}

Find the average of the function
\[f(x)=\sqrt{x}\qquad\text{over}\qquad [0,4].\]

\end{exercise}
\vspace*{6\baselineskip}


\begin{exercise}

Evaluate the integral \[\int_{-2}^2 x\sqrt{x^4+1}\,dx.\]

\end{exercise}

