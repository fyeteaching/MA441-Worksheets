% !TeX root = main.tex

\hypertarget{the-chain-rule}{%
\section{The Chain Rule}\label{the-chain-rule}}

If \(y=f(u)\) and \(u=g(x)\) are differentiable functions, intuitively,
using Leibniz's notation, you may find \[
\frac{\mathrm{d} y}{\mathrm{d} x}=\frac{\mathrm{d} y}{\mathrm{d} u}\cdot\frac{\mathrm{d} u}{\mathrm{d} x}.
\]

This identity is indeed true and called the Chain Rule which is one of
the most important of the differentiation rules.

\begin{theorem}

If \(g\) is differentiable at \(x\) and \(f\) is differentiable at
\(g(x)\), then the composite function \(F=f\circ g\) defined by
\(F(x)=f(g(x))\) is differentiable at \(x\) and \[
F'(x)=f'(g(x))\cdot g'(x).
\]

\end{theorem}

In the chain rule, \(f'(g(x))\) mean the ``output'' of the derivative
function \(f'\) for the ``input'' \(g(x)\).

The theorem can be proved using the error function of an estimation. Let
\(\varepsilon(t)=\frac{f(u+t)-f(u)}{t}-f'(u)\) and \(k=g(x+h)-g(x)\).
Then \[
\begin{aligned}
  \dfrac{f(g(x+h))-f(g(x))}{h}
  =&\dfrac{f(g(x)+k)-f(g(x))}{h}\\
  =&\frac{k(\varepsilon(k)+f'(g(x))}{h}.
\end{aligned}
\] Taking limits will give the chain rule formula.

\begin{example}

\textbf{(General Power Rule):} Let \(f\) be a function differentiable at
\(x\) and \(h(x)=(f(x))^n\). Find \(h'(x)\).

\end{example}
\vspace*{6\baselineskip}

\begin{example}

Find the derivative of \(f(x)=\dfrac{1}{(x+1)^3}\).

\end{example}
\vspace*{6\baselineskip}

\begin{example}

Find the derivative of \(f(x)=\sin(\dfrac{\pi}{2}-x)\).

\end{example}
\vspace*{6\baselineskip}

\begin{example}

Find the derivative of \(f(x)=\tan(\cos x+1)\).

\end{example}
\vspace*{6\baselineskip}

\begin{example}

Find the derivative of \(f(x)=\dfrac{x}{(2x+1)^3}\).

\end{example}
\vspace*{6\baselineskip}

\begin{example}
  Find the derivative
  \(\dfrac{\mathrm{d}}{\mathrm{d}x}\left(\dfrac{1}{\sqrt{1+\sin(x^2+1)}}\right)\).
\end{example}
\vspace*{6\baselineskip}

\begin{example}

Find all points on the curve \(y=\cos x-\cos^2x\) at which the tangent
line is horizontal.

\end{example}
\vspace*{6\baselineskip}

\subsection{Practice}

\begin{exercise}

Find the derivative of \(y=\sqrt{x^2+5}\).

\end{exercise}
\vspace*{6\baselineskip}

\begin{exercise}

Find the derivative \(\dfrac{\mathrm{d}}{\mathrm{d}x}\cos(u+1)\) where
\(u=x^3\).

\end{exercise}
\vspace*{6\baselineskip}

\begin{exercise}

Find the derivatives of the function
\(f(x)=\sqrt{\frac{(x+1)(x-2)}{(x-1)(x+2)}}\).

\end{exercise}
\vspace*{6\baselineskip}

\begin{exercise}

Find all points on the curve \(y=\sqrt{4x+1}\) at which the tangent line
is parallel to the line \(2x-y=1\).

\end{exercise}
\vspace*{6\baselineskip}

