% !TeX root = main.tex

\hypertarget{derivative-functions}{%
\section{Derivative Functions}\label{derivative-functions}}

\hypertarget{derivative-function---definition}{%
\subsection{Derivative Function -
Definition}\label{derivative-function---definition}}

\begin{definition}
Let \(f\) be a function. The \textbf{derivative function}, denoted by
\(f'\), is the function whose domain consists of those values of \(x\)
such that the following limit exists:
\[f'(x)=\lim_{h\to 0}\frac{f(x+h)-f(x)}{h}.\]

A function is said to be \textbf{differentiable over an open set} \(U\)
if it is differentiable at every point in $U$.

A function is called a \textbf{differential function} if it is
differentiable over its domain. At boundary points of the domain, the
differentiability is taken to be the left or right differentiability.
\end{definition}

\begin{example}
Determine whether the function \(f(x)=x^3\) is differentiable and find
the derivative function if it is differentiable.
\end{example}
\vspace*{6\baselineskip}

\begin{example}
Determine whether the function \(f(x)=\sqrt{x}\) is differentiable and
find the derivative function if it is differentiable.
\end{example}
\vspace*{6\baselineskip}

\hypertarget{notations-for-derivatives-and-differentiation}{%
\subsection{Notations for Derivatives and
Differentiation}\label{notations-for-derivatives-and-differentiation}}

When a function is given in the form \(y=f(x)\), we also use \(y'\) to
denote the derivative function. The notations \(f'\) and \(y'\) are
known as Lagrange's ``prime'' notation.

As the derivative function is essentially (the limit) a difference
quotient, we also use \(\frac{\mathrm{d}y}{\mathrm{d x}}\) (or
\(\mathrm{d}y/\mathrm{d x}\)) to denote the derivative functions. The
notation \(\frac{\mathrm{d}y}{\mathrm{d x}}\) was introduced by Leibniz
and called
\href{https://en.wikipedia.org/wiki/Leibniz\%27s_notation}{Leibniz's
notation}.

\begin{remark}
The notation \(\mathrm{d} x\) and \(\mathrm{d} y\) may be considered as
variables and are called \textbf{differentials}. Indeed,
\(\mathrm{d} x=x-a\) is the difference, by \(\mathrm{d}y = f'(a)(x-a)\)
is only an approximation of \(f(x)-f(a)\).
\end{remark}

The process of calculating a derivative is known as
\textbf{differentiation}. We view the prime \('\) or better the notation
\(\frac{\mathrm{d}}{\mathrm{d}x}\) as an operator and call it the
\textbf{differential operator}.

Sometimes, a differential operator is also denoted as \(D\) or \(D_x\)
to indicate the independent variable \(x\). Those notations are known as
Euler's notation.

\hypertarget{differentiability-implies-continuity}{%
\subsection{Differentiability implies
continuity}\label{differentiability-implies-continuity}}

\begin{theorem}
If the function \(f\) is differentiable at \(a\), then \(f\) is
continuous at \(a\).
\end{theorem}

\begin{remark}
\begin{enumerate}[sepno]
\item
  If a function is not continuous, it cannot be differentiable.
\item
  Not every continuous function is differentiable.
\end{enumerate}
\end{remark}

\begin{example}
(Continuous but not differentiable functions)

\begin{enumerate}
\item
  The function \(f(x)=|x|\) is continuous but failed to be
  differentiable at \(0\). Its graph has a sharp corner at \(0.\).
\item
  The function \(f(x)=\sqrt[3]{x}\) also fails to be differentiable at
  \(0\). Because it has a vertical tangent line at \(0\).
\item
  The function
\[f(x)=\begin{cases}x\sin\left(\frac{1}{x}\right),& \text{ if } x\ne 0\\0, & \text{ if } x=\end{cases}\]
is continuous but fails to be differentiable at a point in more
complicated ways.
\end{enumerate}
\end{example}
\vspace*{6\baselineskip}

\begin{example}
Find values of \(a\) and \(b\) that make the following function
differential. \[
f(x)=
\begin{cases}
    ax+ & x\ge 1\\
    x^2-3 & x<1\end{cases}
\]
\end{example}
\vspace*{6\baselineskip}

\hypertarget{higher-derivatives}{%
\subsection{Higher derivatives}\label{higher-derivatives}}

Higher derivatives are defined as repeated differentiations of
functions.

The \textbf{second derivative} \(f''(x)=(f'(x))'\) is defined the
derivative of the first derivative of \(f\).

The \textbf{third derivative} is defined as \(f'''(x)=(f''(x))'\).

The \textbf{\(n\)-th derivative} is defined recursively as
\(f^{(n)}(x)=(f^{(n-1)}(x))'\).

\begin{example}
Find the second derivative \(f''\) of the function \(f(x)=2x^2-3x+1\).
\end{example}
\vspace*{6\baselineskip}

\begin{example}
The position of a particle along a coordinate axis at time \(t\) (in
seconds) is given by \(s(t)=3t^2-4t+1\) (in meters). Find the function
that describes its acceleration at time \(t\).
\end{example}
\vspace*{6\baselineskip}

\subsection{Practice}

\begin{exercise}
  Determine whether the function \(f(x)=x^2-3x\) is differentiable and
  find the derivative function if it is differentiable.
  \end{exercise}
  \vspace*{6\baselineskip}
  
  \begin{exercise}
  Determine whether the function \(f(x)=\sqrt[3]{x}\) is differentiable
  and find the derivative function if it is differentiable.
  \end{exercise}
  \vspace*{6\baselineskip}
  
  \begin{exercise}
  Determine whether the function \(f(x)=\frac{1}{x+1}\) is differentiable
  and find the derivative function if it is differentiable.
  \end{exercise}
  \vspace*{6\baselineskip}
  
\begin{exercise}
Find values of \(a\) and \(b\) that make the following function
differentiable at \(x=3\).
\[f(x)=\begin{cases}ax+b, & \text{ if } x<3\\x^2,& \text{ if } x\ge \end{cases}\]
\end{exercise}
\vspace*{6\baselineskip}

\begin{exercise}
The position of a particle along a coordinate axis at time \(t\) (in
seconds) is given by \(s(t)=2t^3-3t^2+t\) (in meters). Find the function
that describes its acceleration at time \(t\).
\end{exercise}
\vspace*{6\baselineskip}

