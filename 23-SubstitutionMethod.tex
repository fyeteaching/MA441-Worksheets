% !TeX root = main.tex

\hypertarget{substitution-method}{%
\section{Substitution Method}\label{substitution-method}}

Recall rules of derivatives.

\begin{enumerate}[sepno]
\item
  Chain rule: \((f(g(x)))'=f'(g(x))g'(x)\).
\item
  Product Rule: \((f(x)g(x))'=f'(x)g(x)+f(x)g'(x)\).
\end{enumerate}

Reverse those rules, we will get techniques of integration.

\begin{theorem}

\textbf{(Substitution Method for Indefinite Integrals)} Let \(u=g(x)\),
where \(g'(x)\) is continuous over an interval, let \(f(x)\) be
continuous over the corresponding range of g, and let \(F(x)\) be an
antiderivative of \(f(x).\) Then,
\[\int f(g(x))g'(x)\mathrm{d} x =\int f(u)\,)u=F(u)+C= F(g(x))+C.\]

\end{theorem}

\begin{example}

Find the indefinite integral \[\int 2x(x^2+5)^7\mathrm{d} x.\]

\end{example}
\vspace*{6\baselineskip}

\begin{example}

Find the indefinite integral \[\int \sqrt{2x+1}\mathrm{d} x.\]

\end{example}
\vspace*{6\baselineskip}

% \begin{example}

% Find the indefinite integral
% \[\int 8(x+1)\sqrt[3]{x^2+2x}\mathrm{d} x.\]

% \end{example}
% \vspace*{6\baselineskip}

\begin{example}

Find the indefinite integral
\[\int \dfrac{1}{\sqrt{x}(\sqrt{x}+1)}\mathrm{d} x.\]

\end{example}
\vspace*{6\baselineskip}

% \begin{example}

% Find the indefinite integral
% \[\int \dfrac{\sin x}{\cos^5x}\mathrm{d} x.\]

% \end{example}
% \vspace*{6\baselineskip}

\begin{example}

Find the indefinite integral \[\int \cos x(2\sin x+3)^3\mathrm{d} x.\]

\end{example}
\vspace*{6\baselineskip}

\begin{example}

Find the indefinite integral
\[\int \frac{x+1}{\sqrt{x-1}}\mathrm{d} x.\]

\end{example}
\vspace*{6\baselineskip}

\begin{theorem}

\textbf{Substitution for Definite Integral}

Let \(u=g(x)\) and let \(g'\) be continuous over an interval \([a,b]\),
and let \(f\) be continuous over the range of \(u=g(x).\) Then,
\[\int^b_af(g(x))g'(x)\mathrm{d} x=\int^{g(b)}_{g(a)}f(u)\mathrm{d} u.\]

\end{theorem}

\begin{example}

Evaluate the definite integral
\[\int^1_0 x\sin\left(\dfrac{\pi x^2}{2}\right)\mathrm{d} x.\]

\end{example}
\vspace*{6\baselineskip}

\begin{example}

Evaluate the definite integral \[\int^2_1 \cos(3x-2)\mathrm{d} x.\]

\end{example}
\vspace*{6\baselineskip}

% \begin{example}

% Evaluate the definite integral
% \[\int^1_{-2} \dfrac{x}{\sqrt{x^2+1}}\mathrm{d} x.\]

% \end{example}
% \vspace*{6\baselineskip}

% \begin{example}

% Evaluate the definite integral
% \[\displaystyle \int^{3\pi/4}_{\pi/4}\sin^2 t\cos t\mathrm{d} t.\]

% \end{example}
% \vspace*{6\baselineskip}

\subsection{Practice}

\begin{exercise}

Evaluate the following integrals.

\begin{enumerate}
\item
  \(\displaystyle \int x\sqrt{x+1}\mathrm{d} x\)
\item
  \(\displaystyle \int\frac{x}{(4x^2+9)^2}\mathrm{d} x\)
\item
  \(\displaystyle \int(x - 2)^7\mathrm{d} x\)
\item
  \(\displaystyle \int\cos^3 \theta\sin \theta\mathrm{d} \theta\)
\end{enumerate}

\end{exercise}

\begin{exercise}

\begin{enumerate}
\item
  \(\displaystyle \int^1_0x\sqrt{1 - x^2}\mathrm{d} x\)
\item
  \(\displaystyle \int^{\pi/4}_0\sec^2 \theta \tan \theta \mathrm{d} \theta \)
\item
  \(\displaystyle \int^{\pi/4}_0\frac{\sin \theta }{\cos^4 \theta }\mathrm{d} \theta \)
\end{enumerate}

\end{exercise}

